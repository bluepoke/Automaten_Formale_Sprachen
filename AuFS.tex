
\documentclass[18pt,aspectratio=169]{beamer}
%\usetheme{kit}
\usepackage{ngerman}

% metropolis theme: https://github.com/matze/mtheme
\usetheme[sectionpage=progressbar,subsectionpage=progressbar,progressbar=frametitle]{metropolis}
\definecolor{myColor}{HTML}{4477AA}
\definecolor{pitchblack}{HTML}{000000}
\setbeamercolor{alerted text}{fg=myColor}
\setbeamercolor{frametitle}{bg=pitchblack, fg=white}

\makeatletter
\setlength{\metropolis@titleseparator@linewidth}{1pt}
\setlength{\metropolis@progressonsectionpage@linewidth}{2pt}
\setlength{\metropolis@progressinheadfoot@linewidth}{2pt}
\makeatother


% uncomment the following line if you want to hide the navigation symbols
%\beamertemplatenavigationsymbolsempty 

\title{Theoretische Informatik I}
\subtitle{Automaten und Formale Sprachen}
\author{Peter Kossek}

\institute{Berufsakademie Sachsen, Staatliche Studienakademie Leipzig} % Deutsch
\date{}

\makeatletter
\@addtoreset{figure}{section}
\@addtoreset{table}{section}
\makeatother

\renewcommand{\thefigure}{\thesection.\arabic{figure}}

\renewcommand{\thetable}{\thesection.\arabic{table}}

\usepackage{nameref}
\makeatletter
\newcommand*{\currentname}{\@currentlabelname}
\makeatother

\begin{document}

\selectlanguage{ngerman}
%title page
\begin{frame}
	\titlepage
\end{frame}

\begin{frame}
	\frametitle{Danksagung}
	Diese Vorlesung basiert maßgeblich auf dem Script von\\
	Prof. Jochen Kripfganz\\
	Dozent für Theoretische Informatik an der BA Leipzig bis 2021
\end{frame}

\AtBeginSection{
	\begin{frame}
		\sectionpage
		\tableofcontents[sectionstyle=hide/hide,subsectionstyle=show/show/hide]
	\end{frame}
}

%table of contents
\begin{frame}
	\frametitle{Gliederung}
	\tableofcontents
\end{frame}

\frame{
	\frametitle{Sprachen im Kontext der Informatik}
	
	\begin{itemize}
		\item \emph{Sprache:} Menge zulässiger Worte über einem Alphabet
		\item \emph{Programm:} String über einem Alphabet von Eingabezeichen (Quellcode) bzw. binär über 0/1
		\item \emph{Compiler:} 
		\begin{itemize}
			\item prüft die Zulässigkeit dieses Strings als Satz (Wort) einer Sprache
			\item generiert Folge von Maschinenbefehlen
		\end{itemize}
		\item \emph{Lexikalische Analyse:} Übersetzer-Modul ("`Scanner"') fasst Symbolfolgen zu Schlüsselworten ("`Token"') zusammen
		\item \emph{Syntaxanalyse:}
		\begin{itemize}
			\item "`Parser"' prüft, ob Folge von Token zulässig ist
			\item Semantik-Prüfung erfolgt dabei nicht
		\end{itemize}
	\end{itemize}
}

\begin{frame}
	\frametitle{Literatur}
	\begin{itemize}
	  \item Grundlage der Vorlesung
		\begin{itemize}
			\item Schöning, U.: \textit{Logik für Informatiker}
			\item Schöning, U.: \textit{Theoretische Informatik - kurz gefasst}
		\end{itemize}
		\item Weitere Literatur
		\begin{itemize}
			\item Vossen, G.; Witt, K. U.: Grundlagen der Theoretischen
		\end{itemize}
	\end{itemize}
\end{frame}

\section{Aussagenlogik}
\begin{frame}
	\frametitle{Aussagenlogik}
	Gliederung
	\begin{itemize}
		\item Aussagenlogische Formeln
		\item Mengen/Relationen
		\item Boolesche Rechenregeln
		\item Beweistechniken
		\item Erfüllbarkeit aussagenlogischer Formeln
	\end{itemize}
\end{frame}

\begin{frame}
	\frametitle{Aussagenlogik: Syntax}
	\begin{itemize}
		\item \emph{Aussage:} Satz, der entweder wahr ($w$) oder falsch ($f$) ist; Aussagenvariable $A$; Wahrheitswert $w(A)$
		\item \emph{Syntax:} Induktive Definition korrekt gebildeter aussagenlogischer Formeln F über Variablenmenge $V=\{A, B, \ldots\}$:
		\begin{itemize}
			\item Die Booleschen Wahrheitswerte $w$ und $f$ sind Formeln
			\item Jede Variable aus $V$ ist eine Formel: \emph{Atome}
			\item Negation (NICHT): $\neg F$ ist eine Formel
			\item Konjugation (UND): $(F_1 \land F_2)$ ist eine Formel
			\item Disjunktion (ODER): $(F_1 \lor F_2)$ ist eine Formel
			\item Implikation ("`wenn \ldots, dann"'): $(F_1 \rightarrow F_2)$ ist eine Formel
			\item Äquivalenz ("`genau dann, wenn"'): $(F_1 \leftrightarrow F_2)$ ist eine Formel
			\item andere Verknüpfungen bilden keine Formel
		\end{itemize}
	\end{itemize}
\end{frame}

\begin{frame}
	\frametitle{Präzedenzregeln}
	Vereinbarung zur Reduzierung von Klammern
	\begin{itemize}
		\item Bindung (analog "`Punkt vor Strich"'-Rechnung)
		\begin{itemize}
			\item $\neg$ bindet stärker als $\land$
			\item $\land$ bindet stärker als $\lor$
			\item $\lor$ bindet stärker als $\rightarrow$ und $\leftrightarrow$
		\end{itemize}
		\item Operatoren gleicher Stärke: Auswertung linksassoziativ; z.B. \\ $(A \lor B \lor C)$ steht für $((A \lor B) \lor C)$
		\item äußere Klammer weglassen: $(((A \lor B) \rightarrow C) \land B) \mapsto (A \lor B \rightarrow C) \land B$
	\end{itemize}
\end{frame}

\begin{frame}
	\frametitle{Semantik}
	aussagenlogischen Formeln wird eine Bedeutung zugeordnet
	\begin{itemize}
		\item Def. Belegung (Interpretation) $I: V \rightarrow \{f, w\}$\\ den Atomen wird jeweils ein konkreter Wahrheitswert zugeordnet
		\item schrittweise (über die Bewertung von Teilformeln) lassen sich dann zusammengesetzte Formeln bewertn:
		\begin{itemize}
			\item $I(\neg F)=w$ falls $I(F)=f$, sonst $I(\neg F)=f$
			\item $I(F \land G)=w$ falls $I(F)=w$ und $I(G)=w$, sonst $I(F \land G)=f$
			\item $I(F \lor G)=w$ falls $I(F)=w$ oder $I(G)=w$, sonst $I(F \lor G)=f$
			\item $I(F \rightarrow G)=w$ falls $I(F)=f$ oder $I(G)=w$, sonst $I(F \rightarrow G)=f$
			\item $I(F \leftrightarrow G)=w$ falls $I(F)=I(G)$, sonst $I(F \leftrightarrow G)=f$
		\end{itemize}
	\end{itemize}
\end{frame}

\begin{frame}
	\frametitle{Semantik: Darstellung per Wahrheitstafel}
	Wahrheitstafel enthält zeilenweise alle möglichen Belegungen
	\begin{center}
			\begin{tabular}{|c|c|c|c|c|c|c|}
\hline
$A$ & $B$ & $\neg A$ & $A \land B$ & $A \lor B$ & $A \rightarrow B$ & $A \leftrightarrow B$ \\
\hline
f & f & w & f & f & w & w \\
\hline
f & w & w & f & w & w & f \\
\hline
w & f & f & f & w & f & f \\
\hline
w & w & f & w & w & w & w \\
\hline
			\end{tabular}
		\end{center}
		logische Äquivalenz $F_1 \equiv F_2$: $F_1$ und $F_2$ haben gleichen Wahrheitswerteverlauf; z.B.\\
		$A \rightarrow B \equiv \neg A \lor B$\\
		$A \leftrightarrow B \equiv (A \rightarrow B) \land (B \rightarrow A)$
\end{frame}

\begin{frame}
	\frametitle{References}
	\bibliography{example}
%	\bibliographystyle{plain} %does not render "url" fields...
	\bibliographystyle{IEEEtran}  %does render "url" fields, requires "_"s and "#"s to be escaped, e.g. "\_".
\end{frame}

\end{document}

