\documentclass[a4paper,12pt]{article}
\usepackage{fancyhdr}
\usepackage{fancyheadings}
\usepackage[ngerman,german]{babel}
\usepackage{german}
\usepackage[utf8]{inputenc}
%\usepackage[latin1]{inputenc}
\usepackage[active]{srcltx}
\usepackage{algorithm}
\usepackage[noend]{algorithmic}
\usepackage{amsmath}
\usepackage{amssymb}
\usepackage{amsthm}
\usepackage{bbm}
\usepackage{enumerate}
\usepackage{graphicx}
\usepackage{ifthen}
\usepackage{listings}
\usepackage{struktex}
\usepackage{hyperref}

% table placement
\usepackage{placeins}

\newcommand{\Fach}{Theoretische Informatik I - Automaten und Formale Sprachen}
\newcommand{\Name}{}
\newcommand{\Uebungsthema}{Turingmaschinen und Parser}
\newcommand{\Uebungstitel}{Vertiefung in der Praxisphase} %  <-- UPDATE ME

\setlength{\parindent}{0em}
\topmargin -1.0cm
\oddsidemargin 0cm
\evensidemargin 0cm
\setlength{\textheight}{9.2in}
\setlength{\textwidth}{6.0in}

%%%%%%%%%%%%%%%
%% Aufgaben-COMMAND
\newcommand{\Abschnitt}[1]{
	{
		\vspace*{0.5cm}
		\textsf{\textbf{#1}}
		\vspace*{0.2cm}
		
	}
}
%%%%%%%%%%%%%%
\hypersetup{
	pdftitle={\Fach{}: \Uebungstitel{} - \Uebungsthema},
	pdfauthor={\Name},
	pdfborder={0 0 0}
}

\lstset{ %
	language=java,
	basicstyle=\footnotesize\tt,
	showtabs=false,
	tabsize=2,
	captionpos=b,
	breaklines=true,
	extendedchars=true,
	showstringspaces=false,
	flexiblecolumns=true,
}

\title{\Uebungstitel}
\author{\Name{}}

\begin{document}
	\thispagestyle{fancy}
	\chead{\sf \large \Fach{} \\ \small \Name{}}
	\vspace*{0.2cm}
	\begin{center}
		\LARGE \sf \textbf{\Uebungstitel} \\
		\vspace*{0.4cm}
		\Large \sf \textbf{\Uebungsthema}\\
		\vspace*{0.4cm}
	\end{center}
	\vspace*{0.2cm}
	
	%%%%%%%%%%%%%%%%%%%%%%%%%%%%%%%%%%%%
	%% Aufgaben %%%%%%%%%%%%%%%%%%%%%%%%
	%%%%%%%%%%%%%%%%%%%%%%%%%%%%%%%%%%%%
	
	\Abschnitt{Aufgabe 1}
	
	In der Vorlesung wurde besprochen, wie eine 2-Band-Turingmaschine das Wortproblem für die Sprache
	$$L=\{ww \mid w \in \{a,b\}^+\}$$ entscheidet. Da Mehrband-Turingmaschinen und Ein-Band-Turingmaschinen gleich mächtig sind, muss dieses Problem auch von einer Ein-Band-Turingmaschine entschieden werden können. Konstruieren Sie eine solche Turingmaschine.
	
	\Abschnitt{Vorbemerkungen zu Aufgabe 2}
	
	Parser wurden in der Vorlesung nicht thematisiert. Erarbeiten Sie sich die Inhalte des Papers \emph{LR(k)-Analyse für Pragmatiker} von Andreas Kunert, welches unter\\ \url{https://amor.cms.hu-berlin.de/~kunert/papers/lr-analyse/lr.pdf}\\ zur Verfügung steht. Sie sollten sich mindestens die Kapitel 1 bis 3 aneignen, um die folgenden Aufgabe lösen zu können.
	
	\Abschnitt{Aufgabe 2}
	
	Die präfixfreie Variante der Sprache $L$ der korrekt geklammerten Ausdrücke
	$$L=\{(), (()), ((())), (()()), \ldots\}$$
	enthält alle Worte $x$ mit gleicher Anzahl öffnender und schließender Klammern, wobei für jede echte Vorsilbe $u$ (mit $x=uy, y\neq\varepsilon$) die Zahl der öffnenden Klammern größer ist als die Zahl der schließenden Klammern.
	
	\begin{enumerate}[a)]
		\item Entwickeln Sie für $L$ einen deterministischen PDA $A$, der $L$ mittels leerem Keller akzeptiert. \emph{Hinweis:} Dafür ist nur ein einziger Zustand erforderlich.
		\item \emph{Berechnen} Sie aus dem PDA $A$ eine Typ-2-Grammatik $G$, die $L$ produziert.
		\item Prüfen Sie, ob es sich bei $G$ um eine $LR(0)$-Grammatik handelt. Ermitteln Sie dazu den Zustandsübergangsgraph sowie die Parse-Tabelle.
	\end{enumerate}
	
	
\end{document}

