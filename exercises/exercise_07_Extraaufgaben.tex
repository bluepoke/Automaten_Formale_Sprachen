\documentclass[a4paper,12pt]{article}
\usepackage{fancyhdr}
\usepackage{fancyheadings}
\usepackage[ngerman,german]{babel}
\usepackage{german}
\usepackage[utf8]{inputenc}
%\usepackage[latin1]{inputenc}
\usepackage[active]{srcltx}
\usepackage{algorithm}
\usepackage[noend]{algorithmic}
\usepackage{amsmath}
\usepackage{amssymb}
\usepackage{amsthm}
\usepackage{bbm}
\usepackage{enumerate}
\usepackage{graphicx}
\usepackage{ifthen}
\usepackage{listings}
\usepackage{hyperref}
\usepackage{tikz}
\usetikzlibrary{automata, positioning, arrows}

% table placement
\usepackage{placeins}

\newcommand{\Fach}{Theoretische Informatik I - Automaten und Formale Sprachen}
\newcommand{\Name}{}
\newcommand{\Seminargruppe}{5CS24-2} %  <-- UPDATE ME
\newcommand{\Uebung}{7} %  <-- UPDATE ME
\newcommand{\Uebungstitel}{Klausurvorbereitung - Teil 2} %  <-- UPDATE ME
\newcommand{\Besprechungstermin}{12.03.2025} %  <-- UPDATE ME

\setlength{\parindent}{0em}
\topmargin -1.0cm
\oddsidemargin 0cm
\evensidemargin 0cm
\setlength{\textheight}{9.2in}
\setlength{\textwidth}{6.0in}

%%%%%%%%%%%%%%%
%% Aufgaben-COMMAND
\newcommand{\Aufgabe}[1]{
	{
		\vspace*{1cm}
		\textsf{\textbf{Aufgabe: #1}}
		\vspace*{0.2cm}
		
	}
}
%%%%%%%%%%%%%%
\hypersetup{
	pdftitle={\Fach{}: Übungsblatt \Uebung{}},
	pdfauthor={\Name},
	pdfborder={0 0 0}
}

\lstset{ %
	language=java,
	basicstyle=\footnotesize\tt,
	showtabs=false,
	tabsize=2,
	captionpos=b,
	breaklines=true,
	extendedchars=true,
	showstringspaces=false,
	flexiblecolumns=true,
}

\title{Übungsblatt \Uebung{}}
\author{\Name{}}

\begin{document}
	\thispagestyle{fancy}
	\chead{\sf \large \Fach{} \\ \small \Name{}}
	\chead{\sf \large \Fach{} \\ \small Seminargruppe \Seminargruppe{}}
	\vspace*{0.2cm}
	\begin{center}
		\LARGE \sf \textbf{Übung \Uebung{}} \\
		\vspace*{0.4cm}
		\Large \sf \textbf{\Uebungstitel}\\
		\vspace*{0.4cm}
		\normalsize \rm Besprechungstermin: \Besprechungstermin
	\end{center}
	\vspace*{0.2cm}
	
	\tikzset {
		->, % makes the edges directed
		>=stealth, % makes the arrow heads bold
		node distance=1cm and 1cm, % specifies the minimum distance between two nodes.
		every state/.style={thick, fill=gray!10}, %properties for every 'state' node
		initial text=$ $, % Remove text from start arrow
	}

	Gewünscht waren folgende Themen:
	
	\begin{itemize}
		\item Pumping-Lemmas
		\item reguläre Ausdrücke
		\item PDA
		\item TM
	\end{itemize}
	
	%%%%%%%%%%%%%%%%%%%%%%%%%%%%%%%%%%%%
	%% Aufgaben %%%%%%%%%%%%%%%%%%%%%%%%
	%%%%%%%%%%%%%%%%%%%%%%%%%%%%%%%%%%%%
	
	\pagebreak
	
%	\Aufgabe{Hornformel}
%	
%	Gegeben ist die folgende Hornformel:
%	$$F=(A \lor \neg C) \land (\neg A \lor \neg C \lor \neg B) \land C \land (\neg C \lor B)$$
%	Prüfen Sie die Erfüllbarkeit von $F$ jeweils mittels Hornklauselalgorithmus und Markierungsalgorithmus! Notieren Sie die Schritte dabei nachvollziehbar!
%	
%		\Aufgabe{DFA minimieren}
%	
%	Minimieren Sie den folgenden DFA:\\
%	
%	\begin{tikzpicture}
%		\node[state, initial] (s1) {$s_1$};
%		\node[state, above right=of s1] (s2) {$s_2$};
%		\node[state, accepting, right=2cm of s2] (s3) {$s_3$};
%		\node[state, accepting, below right=of s1] (s4) {$s_4$};
%		\node[state, right=2cm of s4] (s5) {$s_5$};
%		\node[state, above right=of s5] (s6) {$s_6$};
%		
%		\draw (s1) edge[above left] node{a} (s2)
%		(s1) edge[below left] node{b} (s4)
%		(s2) edge[above] node{a} (s3)
%		(s2) edge[left] node{b} (s4)
%		(s3) edge[below right, bend left] node{a} (s4)
%		(s3) edge[right] node{b} (s5)
%		(s4) edge[above left, bend left] node{a} (s3)
%		(s4) edge[below] node{b} (s5)
%		(s5) edge[loop right] node{a,b} (s5)
%		(s6) edge[above right] node{a} (s3)
%		(s6) edge[below right] node{b} (s5);
%	\end{tikzpicture}
	
	\Aufgabe{Regulärer Ausdruck}
	
	Geben Sie einen \emph{regulären Ausdruck} für folgende verbal beschriebene Sprache an:
	\begin{itemize}
		\item Alle Wörter der Sprache bestehen nur aus den Zeichen $0$ und $1$.
		\item Das leere Wort gehört nicht zur Sprache.
		\item Die Länge aller Wörter ist gerade.
	\end{itemize}

%	\pagebreak
	
	\Aufgabe{Pumping-Lemma für reguläre Sprachen}
	
	Zeigen Sie, dass
	$$L=\{a^ib^jv \mid v \in \{a\}^* \land |v| \textrm{ ist ungerade } \land j=2i\}$$
	über dem Alphabet $\Sigma=\{a,b\}$ nicht regulär ist.
	
	\Aufgabe{Pumping-Lemma für kontextfreie Sprachen}
	
	Zeigen Sie, dass
	$$L=\{a^ib^jc^k \mid i<j<k\}$$
	über dem Alphabet $\Sigma=\{a,b,c\}$ nicht kontextfrei ist.
	
%	\Aufgabe{Chomsky-Normalform}
%	
%	Geben Sie die folgende Grammatik in der Chomsky-Normalform an:
%	\begin{align*}
%		S \rightarrow & a \mid bA\\
%		A \rightarrow & AB \mid bDA \mid B\\
%		B \rightarrow & a \mid b \mid bDE\\
%		C \rightarrow & AAA \mid \varepsilon\\	
%		D \rightarrow & bAC	
%	\end{align*}

%	\Aufgabe{CYK-Algorithmus}
%	
%	Gegeben sei die Grammatik $G=(\{S, A, B, C, D\}, \{a, b, c\}, S, P)$ mit
%	\begin{align*}
%		P:S\rightarrow & BB \mid AA\\
%		A\rightarrow & AD \mid AC \mid a\\
%		B\rightarrow & BA \mid CB \mid b\\
%		C\rightarrow & BC \mid c\\
%		D\rightarrow & AC \mid b
%	\end{align*}
%	Prüfen Sie mit dem CYK-Algorithmus, ob das Wort $babcba$ in $L(G)$ enthalten ist.

%	\pagebreak
	
	\Aufgabe{PDA}
	
	Die folgende Sprache über dem Alphabet $\Sigma=\{a, b, c\}$ ist deterministisch kontextfrei:
	$$L = \left\{w_1w_2 \in \Sigma^* \mid w_1 \in \{a\}^* \land w_2 \in \{b, c\}^* \land |w_1|=|w_2|\right\}$$
		
	\begin{enumerate}[a)]
		\item Konstruieren Sie einen deterministischen Kellerautomaten, der diese Sprache akzeptiert.
		\item Ist die Akzeptanz mittels leerem Keller möglich? Begründen Sie Ihre Antwort.
	\end{enumerate}

	\pagebreak

	\Aufgabe{Turingmaschine}
	
	Konstruieren Sie eine Turingmaschine, die auf einem Eingabewort aus dem Alphabet $\Sigma = \{|, -\}$ eine Subtraktion $x-y$ durchführt. Dabei stehen Minuend $x$ und Subtrahend $y$ in Bierdeckelnotation hintereinander auf dem Band, getrennt vom Subtraktionsoperator $-$. Nach Abarbeitung des Programms soll nur noch das Ergebnis in Bierdeckelnotation auf dem Band stehen.
	
	Gehen Sie davon aus, dass das Ergebnis der Operation immer größer Null ist, d.h. der Minuend $x$ ist bei jeder Eingabe größer als der Subtrahend $y$.
	
	\emph{Beispiel:} Das Eingabewort für die Operation $4-2$ ist auf dem Band notiert als:
	$$\#||||-||\#$$
	Nach Bearbeitung des Eingabewortes steht auf dem Band
	$$\#||\#$$
	und die Turingmaschine befindet sich in einem Endzustand.
	
%	\pagebreak
%	
%	\Aufgabe{DFA minimieren (encore une fois)}
%	
%	Minimieren Sie den folgenden DFA auf nachvollziehbare Weise:\\
%	
%	\begin{tikzpicture}
%		\node[state, initial] (s0) {$s_0$};
%		\node[state, above right=of s0] (s1) {$s_1$};
%		\node[state, below right=of s0] (s2) {$s_2$};
%		\node[state, accepting, right=2cm of s1] (s3) {$s_3$};
%		\node[state, accepting, right=2cm of s2] (s4) {$s_4$};
%		\node[state, accepting, below right=of s3] (s5) {$s_5$};
%		
%		\draw
%		(s0) edge[above left] node{0} (s1)
%		(s0) edge[below, bend right=75] node{1} (s4)
%		(s1) edge[loop below] node{0} (s1)
%		(s1) edge[above] node{1} (s3)
%		(s2) edge[above right] node{0} (s0)
%		(s2) edge[above left] node{1} (s3)
%		(s3) edge[left] node{0} (s4)
%		(s3) edge[above, bend right=75] node{1} (s0)
%		(s4) edge[loop below] node{0} (s4)
%		(s4) edge[above] node{1} (s2)
%		(s5) edge[loop right] node{0} (s5)
%		(s5) edge[below right] node{1} (s4);
%	\end{tikzpicture}
	
\end{document}

