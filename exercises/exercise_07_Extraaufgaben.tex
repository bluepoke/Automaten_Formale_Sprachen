\documentclass[a4paper,12pt]{article}
\usepackage{fancyhdr}
\usepackage{fancyheadings}
\usepackage[ngerman,german]{babel}
\usepackage{german}
\usepackage[utf8]{inputenc}
%\usepackage[latin1]{inputenc}
\usepackage[active]{srcltx}
\usepackage{algorithm}
\usepackage[noend]{algorithmic}
\usepackage{amsmath}
\usepackage{amssymb}
\usepackage{amsthm}
\usepackage{bbm}
\usepackage{enumerate}
\usepackage{graphicx}
\usepackage{ifthen}
\usepackage{listings}
\usepackage{struktex}
\usepackage{hyperref}
\usepackage{tikz}
\usetikzlibrary{automata, positioning, arrows}

% table placement
\usepackage{placeins}

\newcommand{\Fach}{Theoretische Informatik I - Automaten und Formale Sprachen}
\newcommand{\Name}{}
\newcommand{\Seminargruppe}{5CS21-2} %  <-- UPDATE ME
\newcommand{\Uebung}{7} %  <-- UPDATE ME
\newcommand{\Uebungstitel}{Extraaufgaben} %  <-- UPDATE ME
\newcommand{\Besprechungstermin}{11.03.2022} %  <-- UPDATE ME

\setlength{\parindent}{0em}
\topmargin -1.0cm
\oddsidemargin 0cm
\evensidemargin 0cm
\setlength{\textheight}{9.2in}
\setlength{\textwidth}{6.0in}

%%%%%%%%%%%%%%%
%% Aufgaben-COMMAND
\newcommand{\Aufgabe}[1]{
	{
		\vspace*{0.5cm}
		\textsf{\textbf{Aufgabe #1}}
		\vspace*{0.2cm}
		
	}
}
%%%%%%%%%%%%%%
\hypersetup{
	pdftitle={\Fach{}: Übungsblatt \Uebung{}},
	pdfauthor={\Name},
	pdfborder={0 0 0}
}

\lstset{ %
	language=java,
	basicstyle=\footnotesize\tt,
	showtabs=false,
	tabsize=2,
	captionpos=b,
	breaklines=true,
	extendedchars=true,
	showstringspaces=false,
	flexiblecolumns=true,
}

\title{Übungsblatt \Uebung{}}
\author{\Name{}}

\begin{document}
	\thispagestyle{fancy}
	\chead{\sf \large \Fach{} \\ \small \Name{}}
	\chead{\sf \large \Fach{} \\ \small Seminargruppe \Seminargruppe{}}
	\vspace*{0.2cm}
	\begin{center}
		\LARGE \sf \textbf{Übung \Uebung{}} \\
		\vspace*{0.4cm}
		\Large \sf \textbf{\Uebungstitel}\\
		%\vspace*{0.4cm}
		%\normalsize \rm Besprechungstermin: %\Besprechungstermin
	\end{center}
	\vspace*{0.2cm}
	
	\tikzset {
		->, % makes the edges directed
		>=stealth, % makes the arrow heads bold
		node distance=1cm and 1cm, % specifies the minimum distance between two nodes.
		every state/.style={thick, fill=gray!10}, %properties for every 'state' node
		initial text=$ $, % Remove text from start arrow
	}
	
	%%%%%%%%%%%%%%%%%%%%%%%%%%%%%%%%%%%%
	%% Aufgaben %%%%%%%%%%%%%%%%%%%%%%%%
	%%%%%%%%%%%%%%%%%%%%%%%%%%%%%%%%%%%%
	
	\Aufgabe{Hornformeln}
	
	Gegeben sei folgende Formel:
	$$(\neg A \lor \neg B \lor E) \land C \land A \land (\neg C \lor B) \land (\neg A \lor D) \land (\neg C \lor \neg D \lor A) \land \neg E \land \neg G$$
	Prüfen Sie die Erfüllbarkeit mittels Markierungsalgorithmus und Hornklauselalgorithmus.
	
	\Aufgabe{DFA konstruieren}
	
	$$L_1=\{w \in \{a,b\}^* \mid \textrm{$w$ endet auf $a$ und $|w|$ ist gerade}\}$$
	$$L_2=\{w \in \{a,b\}^* \mid \textrm{$w$ enthält nicht das Teilwort $bb$}\}$$
	
	\Aufgabe{DFA minimieren}
	
	Minimieren Sie den folgenden DFA:
	
	\begin{tikzpicture}
		\node[state, initial] (1) {1};
		\node[state, above right=of 1] (2) {2};
		\node[state, accepting, right=2cm of 2] (3) {3};
		\node[state, accepting, below right=of 1] (4) {4};
		\node[state, right=2cm of 4] (5) {5};
		\node[state, above right=of 5] (6) {6};
		
		\draw (1) edge[above left] node{a} (2)
		      (1) edge[below left] node{b} (4)
		      (2) edge[above] node{a} (3)
		      (2) edge[left] node{b} (4)
		      (3) edge[below right, bend left] node{a} (4)
		      (3) edge[right] node{b} (5)
		      (4) edge[above left, bend left] node{a} (3)
		      (4) edge[below] node{b} (5)
		      (5) edge[loop right] node{a,b} (5)
		      (6) edge[above right] node{a} (3)
		      (6) edge[below right] node{b} (5);
	\end{tikzpicture}
	
	\pagebreak
	
	\Aufgabe{Pumping-Lemma für reguläre Sprachen)}
	
	Zeigen Sie, dass
	$$L=\{a^nb^mv \mid v \in \{a\}^*, |v| \textrm{ ist ungerade und } n=2m\}$$
	nicht regulär ist.
	
	\Aufgabe{CYK-Algorithmus}
	
	Gegeben sei die Grammatik $G=(\{S, A, B, C\}, \{a, b\}, S, P)$ mit
	\begin{align*}
		P:S\rightarrow & BB \mid AA\\
		A\rightarrow & AB \mid AC \mid a\\
		B\rightarrow & BA \mid CB \mid b\\
		C\rightarrow & BC \mid c\\
	\end{align*}
	Prüfen Sie mit dem CYK-Algorithmus, ob das Wort $babcb$ in $L(G)$ enthalten ist.
	
\end{document}

