\documentclass[a4paper,12pt]{article}
\usepackage{fancyhdr}
\usepackage{fancyheadings}
\usepackage[ngerman,german]{babel}
\usepackage{german}
\usepackage[utf8]{inputenc}
%\usepackage[latin1]{inputenc}
\usepackage[active]{srcltx}
\usepackage{algorithm}
\usepackage[noend]{algorithmic}
\usepackage{amsmath}
\usepackage{amssymb}
\usepackage{amsthm}
\usepackage{bbm}
\usepackage{enumerate}
\usepackage{graphicx}
\usepackage{ifthen}
\usepackage{listings}
\usepackage{hyperref}

% table placement
\usepackage{placeins}

\newcommand{\Fach}{Theoretische Informatik I - Automaten und Formale Sprachen}
\newcommand{\Name}{}
\newcommand{\Seminargruppe}{5CS24-2} %  <-- UPDATE ME
\newcommand{\Uebung}{6} %  <-- UPDATE ME
\newcommand{\Uebungstitel}{Klausurvorbereitung} %  <-- UPDATE ME
\newcommand{\Besprechungstermin}{10.03.2025} %  <-- UPDATE ME

\setlength{\parindent}{0em}
\topmargin -1.0cm
\oddsidemargin 0cm
\evensidemargin 0cm
\setlength{\textheight}{9.2in}
\setlength{\textwidth}{6.0in}

%%%%%%%%%%%%%%%
%% Aufgaben-COMMAND
\newcommand{\Aufgabe}[1]{
	{
		\vspace*{0.5cm}
		\textsf{\textbf{Aufgabe #1}}
		\vspace*{0.2cm}
		
	}
}
%%%%%%%%%%%%%%
\hypersetup{
	pdftitle={\Fach{}: Übungsblatt \Uebung{}},
	pdfauthor={\Name},
	pdfborder={0 0 0}
}

\lstset{ %
	language=java,
	basicstyle=\footnotesize\tt,
	showtabs=false,
	tabsize=2,
	captionpos=b,
	breaklines=true,
	extendedchars=true,
	showstringspaces=false,
	flexiblecolumns=true,
}

\title{Übungsblatt \Uebung{}}
\author{\Name{}}

\begin{document}
	\thispagestyle{fancy}
	\chead{\sf \large \Fach{} \\ \small \Name{}}
	\chead{\sf \large \Fach{} \\ \small Seminargruppe \Seminargruppe{}}
	\vspace*{0.2cm}
	\begin{center}
		\LARGE \sf \textbf{Übung \Uebung{}} \\
		\vspace*{0.4cm}
		\Large \sf \textbf{\Uebungstitel}\\
		\vspace*{0.4cm}
		\normalsize \rm Besprechungstermin: \Besprechungstermin
	\end{center}
	\vspace*{0.2cm}
	
	%%%%%%%%%%%%%%%%%%%%%%%%%%%%%%%%%%%%
	%% Aufgaben %%%%%%%%%%%%%%%%%%%%%%%%
	%%%%%%%%%%%%%%%%%%%%%%%%%%%%%%%%%%%%
	
	\Aufgabe{1}
	
	Zeigen Sie mittels Resolution, dass die Formel
	$$F=(\neg P \land \neg Q \land R) \lor (\neg P \land \neg R) \lor (Q \land R) \lor P$$
	eine Tautologie ist.
	
	\Aufgabe{2}
	
	\begin{enumerate}[a)]
		\item Konstruieren Sie einen DFA, der die Wörter über $\Sigma=\{0,1\}$ akzeptiert, die die Zeichenfolge 11010 enthalten.
		\item Konstruieren Sie einen DFA, der folgende Sprache akzeptiert:
		$$L=\{w \in \{a,b\}^* \mid w \textrm{ enthält mindestens ein $a$ und ein $b$}\}$$
	\end{enumerate}
	
	
	\Aufgabe{3}
	
	Gegeben ist folgender Automat:
	\begin{align*}
		A=(&\{s_0, s_1, s_2, s_3, s_4\}, \{0,1\}, \delta, s_0, \{s_3, s_4\})\\
		\delta=\{	&(s_0,0,s_1),(s_0,1,s_2),\\
					&(s_1,0,s_2),(s_1,1,s_3),\\
					&(s_2,0,s_1),(s_2,1,s_3),\\
					&(s_3,0,s_1),(s_3,1,s_4),\\
					&(s_4,0,s_2),(s_4,1,s_4)\}\\
	\end{align*}

	Ermitteln Sie zu $A$ den Minimalautomaten!
	
	\pagebreak
	
	\Aufgabe{4}
	
	Zeigen Sie, dass die Sprache
	$$L=\{a^ib^k \mid k > i\}$$
	nicht regulär ist.
	
	\Aufgabe{5}
	
	Geben Sie die folgende Grammatik in der Chomsky-Normalform an:
	\begin{align*}
		S \rightarrow & a \mid bA\\
		A \rightarrow & AB \mid bDA \mid B\\
		B \rightarrow & a \mid b \mid bDE\\
		C \rightarrow & AAA \mid \varepsilon\\	
		D \rightarrow & bAC	
	\end{align*}
	
	\Aufgabe{6}
	
	Gegeben sei die Grammatik $G=(\{S, A, B, C, D\}, \{a, b, c\}, S, P)$ mit
	\begin{align*}
		P=\{S\rightarrow & BB \mid AA,\\
		A\rightarrow & AD \mid AC \mid a,\\
		B\rightarrow & BA \mid CB \mid b,\\
		C\rightarrow & BC \mid c,\\
		D\rightarrow & AC \mid b\}
	\end{align*}
	Prüfen Sie mit dem CYK-Algorithmus, ob das Wort $babcba$ in $L(G)$ enthalten ist.
	
	\Aufgabe{7}
	
	Gegeben ist folgende Sprache:
	
	$$L=\{w \in \{a,b\}^* \mid |w|_a=|w|_b\}$$
	
	\begin{enumerate}[a)]
		\item Konstruieren Sie einen deterministischen Kellerautomaten, der diese Sprache akzeptiert.
		\item Kann man auch einen DPDA konstruieren, der die Sprache mittels leerem Keller akzeptiert?
	\end{enumerate}
	
	\Aufgabe{8}
	
	Konstruieren Sie eine 1-Band-Turingmaschine, die eine Multiplikation natürlicher Zahlen berechnet. Die Eingabe soll dabei in Bierdeckelnotation erfolgen. Als Beispiel sei die Rechnung $3\cdot2$ genannt. Auf dem Band steht dann zu Beginn: \texttt{\#|||$\cdot$||\#}
	
	Nach der Abarbeitung durch die TM soll auf dem Band das Ergebnis stehen, hier also die Zahl $6$ bzw. \texttt{\#||||||\#}
\end{document}

