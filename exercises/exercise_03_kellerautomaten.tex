\documentclass[a4paper,12pt]{article}
\usepackage{fancyhdr}
\usepackage{fancyheadings}
\usepackage[ngerman,german]{babel}
\usepackage{german}
\usepackage[utf8]{inputenc}
%\usepackage[latin1]{inputenc}
\usepackage[active]{srcltx}
\usepackage{algorithm}
\usepackage[noend]{algorithmic}
\usepackage{amsmath}
\usepackage{amssymb}
\usepackage{amsthm}
\usepackage{bbm}
\usepackage{enumerate}
\usepackage{graphicx}
\usepackage{ifthen}
\usepackage{listings}
\usepackage{hyperref}

% table placement
\usepackage{placeins}

\newcommand{\Fach}{Theoretische Informatik I - Automaten und Formale Sprachen}
\newcommand{\Name}{}
\newcommand{\Uebung}{3} %  <-- UPDATE ME
\newcommand{\Uebungstitel}{Kellerautomaten} 

\setlength{\parindent}{0em}
\topmargin -1.0cm
\oddsidemargin 0cm
\evensidemargin 0cm
\setlength{\textheight}{9.2in}
\setlength{\textwidth}{6.0in}

%%%%%%%%%%%%%%%
%% Aufgaben-COMMAND
\newcommand{\Aufgabe}[1]{
	{
		\vspace*{0.5cm}
		\textsf{\textbf{Aufgabe #1}}
		\vspace*{0.2cm}
		
	}
}
%%%%%%%%%%%%%%
\hypersetup{
	pdftitle={\Fach{}: Übungsblatt \Uebung{}},
	pdfauthor={\Name},
	pdfborder={0 0 0}
}

\lstset{ %
	language=java,
	basicstyle=\footnotesize\tt,
	showtabs=false,
	tabsize=2,
	captionpos=b,
	breaklines=true,
	extendedchars=true,
	showstringspaces=false,
	flexiblecolumns=true,
}

\title{Übungsblatt \Uebung{}}
\author{\Name{}}

\begin{document}
	\thispagestyle{fancy}
	\chead{\sf \large \Fach{}}
	\vspace*{0.2cm}
	\begin{center}
		\LARGE \sf \textbf{Übung \Uebung{}} \\
		\vspace*{0.4cm}
		\Large \sf \textbf{\Uebungstitel}\\
	\end{center}
	\vspace*{0.2cm}
	
	%%%%%%%%%%%%%%%%%%%%%%%%%%%%%%%%%%%%
	%% Aufgaben %%%%%%%%%%%%%%%%%%%%%%%%
	%%%%%%%%%%%%%%%%%%%%%%%%%%%%%%%%%%%%
	
	\Aufgabe{1}
	Sind die folgenden Sprachen kontextfrei, bzw. sogar deterministisch kontextfrei? Begründen Sie Ihre Antwort, gegebenenfalls durch Angabe einer geeigneten Grammatik bzw. eines geeigneten Automaten.
	
	\begin{enumerate}[a)]
		\item $L_1=\{a^ib^jc^k \mid i>j \textrm{ oder } j>k,\quad i,j,k\geq 0\}$
		\item $L_2=\{ww \mid w\in \{a\}^*\}$
		\item $L_3=\{ww \mid w\in \{a,b\}^*\}$
	\end{enumerate}
	
	\Aufgabe{2}
	Konstruieren Sie einen deterministischen PDA, der die Sprache\\$L=\{a^nb^m \mid m \leq 2n, \quad m,n\geq 1\}$ akzeptiert. Ist das mittels leerem Keller möglich?
	
	\Aufgabe{3}
	Konstruieren Sie einen deterministischen PDA, der die Sprache der korrekt geklammerten Ausdrücke\\
	$L=\{w=xy \in \{[,]\}^* \mid \textrm{Anzahl}(w,[)=\textrm{Anzahl}(w,]),$\\
	\qquad und für jede Vorsilbe $x$ gilt: $\textrm{Anzahl}(x,[)\geq\textrm{Anzahl}(x,])\}$\\
	akzeptiert.
	
	Kann das durch Akzeptanz mit leerem Keller erreicht werden? Begründen Sie!
	
	\Aufgabe{4}
	Die Sprache $L=\{a^ib^ia^kb^k \mid i,k\geq 1\}$ ist deterministisch kontextfrei.\\
	Lässt sich diese Sprache durch einen DPDA mit leerem Keller akzeptieren?\\
	Entwickeln Sie einen DPDA mit dem geeigneten Akzeptanzverhalten.
	
\end{document}

